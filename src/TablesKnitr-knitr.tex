\documentclass[11pt]{article}\usepackage[]{graphicx}\usepackage[]{color}
%% maxwidth is the original width if it is less than linewidth
%% otherwise use linewidth (to make sure the graphics do not exceed the margin)
\makeatletter
\def\maxwidth{ %
  \ifdim\Gin@nat@width>\linewidth
    \linewidth
  \else
    \Gin@nat@width
  \fi
}
\makeatother

\definecolor{fgcolor}{rgb}{0.345, 0.345, 0.345}
\newcommand{\hlnum}[1]{\textcolor[rgb]{0.686,0.059,0.569}{#1}}%
\newcommand{\hlstr}[1]{\textcolor[rgb]{0.192,0.494,0.8}{#1}}%
\newcommand{\hlcom}[1]{\textcolor[rgb]{0.678,0.584,0.686}{\textit{#1}}}%
\newcommand{\hlopt}[1]{\textcolor[rgb]{0,0,0}{#1}}%
\newcommand{\hlstd}[1]{\textcolor[rgb]{0.345,0.345,0.345}{#1}}%
\newcommand{\hlkwa}[1]{\textcolor[rgb]{0.161,0.373,0.58}{\textbf{#1}}}%
\newcommand{\hlkwb}[1]{\textcolor[rgb]{0.69,0.353,0.396}{#1}}%
\newcommand{\hlkwc}[1]{\textcolor[rgb]{0.333,0.667,0.333}{#1}}%
\newcommand{\hlkwd}[1]{\textcolor[rgb]{0.737,0.353,0.396}{\textbf{#1}}}%

\usepackage{framed}
\makeatletter
\newenvironment{kframe}{%
 \def\at@end@of@kframe{}%
 \ifinner\ifhmode%
  \def\at@end@of@kframe{\end{minipage}}%
  \begin{minipage}{\columnwidth}%
 \fi\fi%
 \def\FrameCommand##1{\hskip\@totalleftmargin \hskip-\fboxsep
 \colorbox{shadecolor}{##1}\hskip-\fboxsep
     % There is no \\@totalrightmargin, so:
     \hskip-\linewidth \hskip-\@totalleftmargin \hskip\columnwidth}%
 \MakeFramed {\advance\hsize-\width
   \@totalleftmargin\z@ \linewidth\hsize
   \@setminipage}}%
 {\par\unskip\endMakeFramed%
 \at@end@of@kframe}
\makeatother

\definecolor{shadecolor}{rgb}{.97, .97, .97}
\definecolor{messagecolor}{rgb}{0, 0, 0}
\definecolor{warningcolor}{rgb}{1, 0, 1}
\definecolor{errorcolor}{rgb}{1, 0, 0}
\newenvironment{knitrout}{}{} % an empty environment to be redefined in TeX

\usepackage{alltt}
\usepackage[margin=1in]{geometry}   % set up margins
\usepackage{enumerate}              % fancy enumerate
\usepackage{amsmath}                % used for \eqref{} in this document
\usepackage{verbatim}               % useful for \begin{comment} and \end{comment}
\usepackage{comment}
\usepackage[pdftitle={Homework With knitr}, colorlinks=true, linkcolor=blue,
citecolor=blue, urlcolor=blue, linktocpage=true, breaklinks=true]{hyperref}
%%%%%%%%%%%%%%%%%%%%%%%%%%%%%%%%%%%%%%%%%%%%%%%%%%%%%%%%%%%%%%%%%%%%%%
\IfFileExists{upquote.sty}{\usepackage{upquote}}{}
\begin{document}
%%%%%%%%%%%%%%%%   Sweave Options  %%%%%%%%%%%%%%%%%%%%%%%%%%%%%%%%%%%



%%%%%%%%%%%%%%%%%%%%%%%%%%%%%%%%%%%%%%%%%%%%%%%%%%%%%%%%%%%%%%%%%%%%%%
\title{Tables with \textbf{xtable} and \textbf{knitr}}
\author{Alan T. Arnholt\\ STT 3851}
\date{Spring 2012}
\maketitle



Here are a few examples using the function \texttt{xtable()} form the \textbf{R} package \texttt{xtable} I used to generate \LaTeX{} code for tabular output without manually entering the values in a tabular environment. To get beyond the basic examples in this document, read the documentation and customize until you are happy.  Consider a table created with \texttt{xtabs()}.

\begin{knitrout}
\definecolor{shadecolor}{rgb}{0.969, 0.969, 0.969}\color{fgcolor}\begin{kframe}
\begin{alltt}
\hlstd{T1} \hlkwb{<-} \hlkwd{xtabs}\hlstd{(}\hlopt{~} \hlstd{Ease} \hlopt{+} \hlstd{Treatment,} \hlkwc{data} \hlstd{= EPIDURALf)}
\hlstd{T1}
\end{alltt}
\begin{verbatim}
            Treatment
Ease         Hamstring Stretch Traditional Sitting
  Difficult                 63                  51
  Easy                     100                 107
  Impossible                 8                  13
\end{verbatim}
\end{kframe}
\end{knitrout}

Now we want this to appear in a ``pretty'' format.

% latex table generated in R 3.1.2 by xtable 1.7-4 package
% Sun Jan 18 20:52:25 2015
\begin{table}[ht]
\centering
\begin{tabular}{rrr}
  \hline
 & Hamstring Stretch & Traditional Sitting \\ 
  \hline
Difficult & 63.00 & 51.00 \\ 
  Easy & 100.00 & 107.00 \\ 
  Impossible & 8.00 & 13.00 \\ 
   \hline
\end{tabular}
\end{table}


Creating a caption with \texttt{xtable()} for Table \ref{MyT1}.
% latex table generated in R 3.1.2 by xtable 1.7-4 package
% Sun Jan 18 20:52:25 2015
\begin{table}[ht]
\centering
\begin{tabular}{rrr}
  \hline
 & Hamstring Stretch & Traditional Sitting \\ 
  \hline
Difficult & 63.00 & 51.00 \\ 
  Easy & 100.00 & 107.00 \\ 
  Impossible & 8.00 & 13.00 \\ 
   \hline
\end{tabular}
\caption{Table of Something} 
\label{MyT1}
\end{table}


Consider the regression results below and those shown in Tables \ref{RR} and \ref{RR2}.

\begin{knitrout}
\definecolor{shadecolor}{rgb}{0.969, 0.969, 0.969}\color{fgcolor}\begin{kframe}
\begin{alltt}
\hlstd{mod} \hlkwb{<-} \hlkwd{lm}\hlstd{(gpa} \hlopt{~} \hlstd{sat,} \hlkwc{data} \hlstd{= Grades)}
\hlstd{SR} \hlkwb{<-} \hlkwd{summary}\hlstd{(mod)}\hlopt{$}\hlstd{coefficients}
\hlstd{SR}
\end{alltt}
\begin{verbatim}
               Estimate  Std. Error  t value     Pr(>|t|)
(Intercept) -1.19206381 0.222450180 -5.35879 2.316666e-07
sat          0.00309427 0.000194465 15.91171 2.922995e-37
\end{verbatim}
\begin{alltt}
\hlstd{AR} \hlkwb{<-} \hlkwd{anova}\hlstd{(mod)}
\hlstd{AR}
\end{alltt}
\begin{verbatim}
Analysis of Variance Table

Response: gpa
           Df Sum Sq Mean Sq F value    Pr(>F)    
sat         1 40.397  40.397  253.18 < 2.2e-16 ***
Residuals 198 31.592   0.160                      
---
Signif. codes:  0 '***' 0.001 '**' 0.01 '*' 0.05 '.' 0.1 ' ' 1
\end{verbatim}
\end{kframe}
\end{knitrout}

% latex table generated in R 3.1.2 by xtable 1.7-4 package
% Sun Jan 18 20:52:25 2015
\begin{table}[ht]
\centering
\begin{tabular}{rrrrr}
  \hline
 & Estimate & Std. Error & t value & Pr($>$$|$t$|$) \\ 
  \hline
(Intercept) & -1.19 & 0.22 & -5.36 & 0.00 \\ 
  sat & 0.00 & 0.00 & 15.91 & 0.00 \\ 
   \hline
\end{tabular}
\caption{Regression results} 
\label{RR}
\end{table}


% latex table generated in R 3.1.2 by xtable 1.7-4 package
% Sun Jan 18 20:52:25 2015
\begin{table}[ht]
\centering
\begin{tabular}{rrrrr}
  \hline
 & Estimate & Std. Error & t value & Pr($>$$|$t$|$) \\ 
  \hline
(Intercept) & -1.1921 & 0.2225 & -5.3588 & 0.0000 \\ 
  sat & 0.0031 & 0.0002 & 15.9117 & 0.0000 \\ 
   \hline
\end{tabular}
\caption{Regression results with 4 digits} 
\label{RR2}
\end{table}


% latex table generated in R 3.1.2 by xtable 1.7-4 package
% Sun Jan 18 20:52:25 2015
\begin{table}[ht]
\centering
\begin{tabular}{lrrrrr}
  \hline
 & Df & Sum Sq & Mean Sq & F value & Pr($>$F) \\ 
  \hline
sat & 1 & 40.40 & 40.40 & 253.18 & 0.0000 \\ 
  Residuals & 198 & 31.59 & 0.16 &  &  \\ 
   \hline
\end{tabular}
\caption{ANOVA} 
\label{AR}
\end{table}


Suppose you want the label \texttt{Pr(>F)} in Table \ref{AR} to read $\wp$-value.

% latex table generated in R 3.1.2 by xtable 1.7-4 package
% Sun Jan 18 20:52:25 2015
\begin{table}[ht]
\centering
\begin{tabular}{lrrrrr}
  \hline
 & DOF & Sum Sq & Mean Sq & F value & $\wp$-value \\ 
  \hline
sat & 1.00 & 40.40 & 40.40 & 253.18 & 0.00 \\ 
  Residuals & 198.00 & 31.59 & 0.16 &  &  \\ 
   \hline
\end{tabular}
\caption{ANOVA with $\wp$-value changed} 
\label{ARpvalue}
\end{table}


\end{document}
