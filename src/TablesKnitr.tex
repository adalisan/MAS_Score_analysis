\documentclass[11pt]{article}\usepackage[]{graphicx}\usepackage[]{color}
%% maxwidth is the original width if it is less than linewidth
%% otherwise use linewidth (to make sure the graphics do not exceed the margin)
\makeatletter
\def\maxwidth{ %
  \ifdim\Gin@nat@width>\linewidth
    \linewidth
  \else
    \Gin@nat@width
  \fi
}
\makeatother

\definecolor{fgcolor}{rgb}{0.345, 0.345, 0.345}
\newcommand{\hlnum}[1]{\textcolor[rgb]{0.686,0.059,0.569}{#1}}%
\newcommand{\hlstr}[1]{\textcolor[rgb]{0.192,0.494,0.8}{#1}}%
\newcommand{\hlcom}[1]{\textcolor[rgb]{0.678,0.584,0.686}{\textit{#1}}}%
\newcommand{\hlopt}[1]{\textcolor[rgb]{0,0,0}{#1}}%
\newcommand{\hlstd}[1]{\textcolor[rgb]{0.345,0.345,0.345}{#1}}%
\newcommand{\hlkwa}[1]{\textcolor[rgb]{0.161,0.373,0.58}{\textbf{#1}}}%
\newcommand{\hlkwb}[1]{\textcolor[rgb]{0.69,0.353,0.396}{#1}}%
\newcommand{\hlkwc}[1]{\textcolor[rgb]{0.333,0.667,0.333}{#1}}%
\newcommand{\hlkwd}[1]{\textcolor[rgb]{0.737,0.353,0.396}{\textbf{#1}}}%

\usepackage{framed}
\makeatletter
\newenvironment{kframe}{%
 \def\at@end@of@kframe{}%
 \ifinner\ifhmode%
  \def\at@end@of@kframe{\end{minipage}}%
  \begin{minipage}{\columnwidth}%
 \fi\fi%
 \def\FrameCommand##1{\hskip\@totalleftmargin \hskip-\fboxsep
 \colorbox{shadecolor}{##1}\hskip-\fboxsep
     % There is no \\@totalrightmargin, so:
     \hskip-\linewidth \hskip-\@totalleftmargin \hskip\columnwidth}%
 \MakeFramed {\advance\hsize-\width
   \@totalleftmargin\z@ \linewidth\hsize
   \@setminipage}}%
 {\par\unskip\endMakeFramed%
 \at@end@of@kframe}
\makeatother

\definecolor{shadecolor}{rgb}{.97, .97, .97}
\definecolor{messagecolor}{rgb}{0, 0, 0}
\definecolor{warningcolor}{rgb}{1, 0, 1}
\definecolor{errorcolor}{rgb}{1, 0, 0}
\newenvironment{knitrout}{}{} % an empty environment to be redefined in TeX

\usepackage{alltt}
\usepackage[margin=1in]{geometry}   % set up margins
\usepackage{enumerate}              % fancy enumerate
\usepackage{amsmath}                % used for \eqref{} in this document
\usepackage{verbatim}               % useful for \begin{comment} and \end{comment}
\usepackage{comment}
\usepackage[pdftitle={Homework With knitr}, colorlinks=true, linkcolor=blue,
citecolor=blue, urlcolor=blue, linktocpage=true, breaklinks=true]{hyperref}
%%%%%%%%%%%%%%%%%%%%%%%%%%%%%%%%%%%%%%%%%%%%%%%%%%%%%%%%%%%%%%%%%%%%%%
\IfFileExists{upquote.sty}{\usepackage{upquote}}{}
\begin{document}
%%%%%%%%%%%%%%%%   Sweave Options  %%%%%%%%%%%%%%%%%%%%%%%%%%%%%%%%%%%
\SweaveOpts{fig.path='./Graphs/alan-', comment=NA, prompt=FALSE}
%%%%%%%%%%%%%%%%%%%%%%%%%%%%%%%%%%%%%%%%%%%%%%%%%%%%%%%%%%%%%%%%%%%%%%
\title{Tables with \textbf{xtable} and \textbf{knitr}}
\author{Alan T. Arnholt\\ STT 3851}
\date{Spring 2012}
\maketitle

\begin{knitrout}
\definecolor{shadecolor}{rgb}{0.969, 0.969, 0.969}\color{fgcolor}\begin{kframe}


{\ttfamily\noindent\color{warningcolor}{\#\# Warning in library(package, lib.loc = lib.loc, character.only = TRUE, logical.return = TRUE, : there is no package called 'PASWR'}}\end{kframe}
\end{knitrout}

Here are a few examples using the function \texttt{xtable()} form the \textbf{R} package \texttt{xtable} I used to generate \LaTeX{} code for tabular output without manually entering the values in a tabular environment. To get beyond the basic examples in this document, read the documentation and customize until you are happy.  Consider a table created with \texttt{xtabs()}.

\begin{knitrout}
\definecolor{shadecolor}{rgb}{0.969, 0.969, 0.969}\color{fgcolor}\begin{kframe}
\begin{alltt}
\hlstd{T1} \hlkwb{<-} \hlkwd{xtabs}\hlstd{(}\hlopt{~} \hlstd{Ease} \hlopt{+} \hlstd{Treatment,} \hlkwc{data} \hlstd{= EPIDURALf)}
\end{alltt}


{\ttfamily\noindent\bfseries\color{errorcolor}{\#\# Error in terms.formula(formula, data = data): object 'EPIDURALf' not found}}\begin{alltt}
\hlstd{T1}
\end{alltt}


{\ttfamily\noindent\bfseries\color{errorcolor}{\#\# Error in eval(expr, envir, enclos): object 'T1' not found}}\end{kframe}
\end{knitrout}

Now we want this to appear in a ``pretty'' format.

\begin{kframe}


{\ttfamily\noindent\bfseries\color{errorcolor}{\#\# Error in xtable(T1): object 'T1' not found}}\end{kframe}

Creating a caption with \texttt{xtable()} for Table \ref{MyT1}.
\begin{kframe}


{\ttfamily\noindent\bfseries\color{errorcolor}{\#\# Error in xtable(T1, caption = "{}Table of Something"{}, label = "{}MyT1"{}): object 'T1' not found}}\end{kframe}

Consider the regression results below and those shown in Tables \ref{RR} and \ref{RR2}.

\begin{knitrout}
\definecolor{shadecolor}{rgb}{0.969, 0.969, 0.969}\color{fgcolor}\begin{kframe}
\begin{alltt}
\hlstd{mod} \hlkwb{<-} \hlkwd{lm}\hlstd{(gpa} \hlopt{~} \hlstd{sat,} \hlkwc{data} \hlstd{= Grades)}
\end{alltt}


{\ttfamily\noindent\bfseries\color{errorcolor}{\#\# Error in is.data.frame(data): object 'Grades' not found}}\begin{alltt}
\hlstd{SR} \hlkwb{<-} \hlkwd{summary}\hlstd{(mod)}\hlopt{$}\hlstd{coefficients}
\end{alltt}


{\ttfamily\noindent\bfseries\color{errorcolor}{\#\# Error in summary(mod): object 'mod' not found}}\begin{alltt}
\hlstd{SR}
\end{alltt}


{\ttfamily\noindent\bfseries\color{errorcolor}{\#\# Error in eval(expr, envir, enclos): object 'SR' not found}}\begin{alltt}
\hlstd{AR} \hlkwb{<-} \hlkwd{anova}\hlstd{(mod)}
\end{alltt}


{\ttfamily\noindent\bfseries\color{errorcolor}{\#\# Error in anova(mod): object 'mod' not found}}\begin{alltt}
\hlstd{AR}
\end{alltt}


{\ttfamily\noindent\bfseries\color{errorcolor}{\#\# Error in eval(expr, envir, enclos): object 'AR' not found}}\end{kframe}
\end{knitrout}

\begin{kframe}


{\ttfamily\noindent\bfseries\color{errorcolor}{\#\# Error in xtable(SR, caption = "{}Regression results"{}, label = "{}RR"{}): object 'SR' not found}}\end{kframe}

\begin{kframe}


{\ttfamily\noindent\bfseries\color{errorcolor}{\#\# Error in xtable(SR, caption = "{}Regression results with 4 digits"{}, label = "{}RR2"{}, : object 'SR' not found}}\end{kframe}

\begin{kframe}


{\ttfamily\noindent\bfseries\color{errorcolor}{\#\# Error in xtable(AR, caption = "{}ANOVA"{}, label = "{}AR"{}): object 'AR' not found}}\end{kframe}

Suppose you want the label \texttt{Pr(>F)} in Table \ref{AR} to read $\wp$-value.

\begin{kframe}


{\ttfamily\noindent\bfseries\color{errorcolor}{\#\# Error in colnames(AR) <- c("{}DOF"{}, "{}Sum Sq"{}, "{}Mean Sq"{}, "{}F value"{}, "{}\$\textbackslash{}\textbackslash{}wp\$-value"{}): object 'AR' not found}}

{\ttfamily\noindent\bfseries\color{errorcolor}{\#\# Error in xtable(AR, caption = "{}ANOVA with \$\textbackslash{}\textbackslash{}wp\$-value changed"{}, label = "{}ARpvalue"{}): object 'AR' not found}}

{\ttfamily\noindent\bfseries\color{errorcolor}{\#\# Error in print(AR, sanitize.text.function = function(x) \{: object 'AR' not found}}\end{kframe}

\end{document}
